% \iffalse meta-comment
% Droits d'auteur : Maïeul Rouquette 2011
% Licence Creative Commons - Paternité -Partage des Conditions Initiales à l'identique 
% http://creativecommons.org/licenses/by-sa/2.0/fr/
% \fi
% \iffalse
%<*driver>
\ProvidesFile{facture.dtx}
\documentclass{ltxdoc}
\EnableCrossrefs
\RecordChanges
\begin{document}
  \DocInput{facture.dtx}
\end{document}
%</driver>
%\fi
% \CharacterTable
%  {Upper-case    \A\B\C\D\E\F\G\H\I\J\K\L\M\N\O\P\Q\R\S\T\U\V\W\X\Y\Z
%   Lower-case    \a\b\c\d\e\f\g\h\i\j\k\l\m\n\o\p\q\r\s\t\u\v\w\x\y\z
%   Digits        \0\1\2\3\4\5\6\7\8\9
%   Exclamation   \!     Double quote  \"     Hash (number) \#
%   Dollar        \$     Percent       \%     Ampersand     \&
%   Acute accent  \'     Left paren    \(     Right paren   \)
%   Asterisk      \*     Plus          \+     Comma         \,
%   Minus         \-     Point         \.     Solidus       \/
%   Colon         \:     Semicolon     \;     Less than     \<
%   Equals        \=     Greater than  \>     Question mark \?
%   Commercial at \@     Left bracket  \[     Backslash     \\
%   Right bracket \]     Circumflex    \^     Underscore    \_
%   Grave accent  \`     Left brace    \{     Vertical bar  \|
%   Right brace   \}     Tilde         \~}
% \CheckSum{0}
% \changes{v1.0}{2011/09/10}{Première version}
% \GetFileInfo{facture.dtx}
% \title{La class \textsf{facture}\thanks{Ce document correspond à la version~\fileversion de \textsf{facture},    datée du~\filedate.}}
%\author{Maïeul Rouquette}
%\maketitle
%\begin{abstract}
%Cette classe permet de rédiger facture et devis, avec ou sans TVA, en calculant automatiquement les sommes.
%\end{abstract}
%    \begin{macrocode}
%    \end{macrocode} 
%\begin{macrocode}
\NeedsTeXFormat{LaTeX2e}
\ProvidesClass{facture}[2011/09/10 v1.0]
\RequirePackage{fontspec}    
\RequirePackage{xunicode}
\RequirePackage{polyglossia}
\setmainlanguage{french}
\RequirePackage{numprint}
\RequirePackage{fltpoint}% Pour faire les calculs dans le tableau
\RequirePackage{tikz} % tikz est utilise pour tracer des boites, par exemple
\RequirePackage{graphicx} % Pour insérer des images. Utiliser le format jpg pour plus de simplicité.
\RequirePackage{fancyhdr} % Pour entête et pied de page
\RequirePackage{pifont} % Pour utiliser des symboles divers.
\RequirePackage[paper=a4paper,top=2 cm, bottom=2 cm, left=1.5 cm, right=2.5 cm]{geometry} % On peut modifier ces valeurs pour augmenter ou réduire les marges. 
\RequirePackage{ifthen}
\RequirePackage{array}
\RequirePackage{longtable}
\RequirePackage{advdate}
\RequirePackage{etoolbox}
\RequirePackage{xargs}
\RequirePackage{colortbl}
%\end{macrocode}
\newif\ifsansTVA
\sansTVAfalse
\DeclareOption{sansTVA}{\sansTVAtrue}
\ProcessOptions

\definecolor{entetes}{HTML}{888888}
\definecolor{encadre}{RGB}{111,111,111}
\definecolor{blanc}{RGB}{255,255,255}
\newcommand{\libelle}[1]{\textcolor{entetes}{\textbf#1}}



\newcommand{\codeclient}{Code client :}
\newcommand{\datetxt}{Date:}
\newcommand{\datelimitetxt}{À payer avant:}
\newcommand{\facturation}{Facturation}
\newcommand{\facturetxt}{Facture}
\newcommand{\livraison}{Livraison}
\newcommand{\livraisonfacturation}{Livraison et Facturation}
\newcommand{\ntxt}{~n°}

\newcommand{\produit}{Produit}
\newcommand{\quantite}{Quantité}
\newcommand{\remise}{Remise}
\newcommand{\unite}{€}

\ifsansTVA%
    \newcommand{\tottxt}{Total}
    \newcommand{\prixtxt}{Prix}
    \newcommand{\postTotaux}{\hfill TVA non applicable, art. 293 B}
\else%
    \newcommand{\prixHT}{Prix HT}
    \newcommand{\TVAtxt}{TVA}
    \newcommand{\prixTTC}{Prix TTC}
    \newcommand{\totHTtxt}{Total HT}
    \newcommand{\totTVAtxt}{Total TVA}
    \newcommand{\totTTCtxt}{Total TTC}
    \newcommand{\postTotaux}{}
\fi


\npaddmissingzero
\nprounddigits{2}
\newcommand{\tvadefaut}{19,6}


\gdef\@type{\facturetxt}

\newcommand{\datelimite}[1]{\newcommand{\@datelimite}{#1}}
\newcommand{\dest}[1]{\newcommand{\@dest}{#1}}
\newcommand{\fact}[1]{\newcommand{\@fact}{#1}}
\newcommand{\adresseemet}[1]{\gdef\@adresseemet{#1}}
\newcommand{\nomemet}[1]{\gdef\@nomemet{#1}}
\newcommand{\type}[1]{\gdef\@type{#1}}
\newcommand{\numero}[1]{\gdef\@numero{#1}}
\newcommand{\codedest}[1]{\gdef\@codedest{#1}}
\newcommand{\pied}[1]{\rfoot{#1}}
\setlength{\tabcolsep}{1pt}


\newcommand{\entete}{
    
    
    \noindent\begin{tikzpicture}
    
        \noindent\node [font=\bf\Huge,text width=0.5\textwidth,text=entetes,text centered]{%        
            \@nomemet%
        };
            
        \noindent\node (b)[xshift=0.5\textwidth,text width=0.5\textwidth, right]{%
            \@adresseemet%
        };
    
    % Détermination des coordonnées du rectangle, en fonction de l'adresse
    \coordinate[xshift=-1\textwidth,yshift=1ex] (a) at (b.north);
    \coordinate[xshift=1em,yshift=-1ex] (c) at (b.south);
    \draw[color=encadre,line width=1.5mm] (a) rectangle  (c);

%    \draw (b.north west) rectangle (b) ;
    \end{tikzpicture}
        
    
    \vspace{3ex}    
    
    \begin{flushright}    
        
    
    {\Huge\libelle{\@type}}
    
    \end{flushright}    
    
    \vspace{3ex}    
    
    \begin{tikzpicture}
    

    \ifdef{\@fact}{
    \node[text width=0.33\textwidth,anchor=base]{
        \libelle{\facturation}
        
        \@fact
        };
    }{}
    
    \node[xshift=0.33\textwidth,text width=0.33\textwidth,anchor=base]{\ifdef{\@fact}{
            \libelle{\livraison}}
            {\libelle{\livraisonfacturation}}
            
        \@dest
    };
    \node[text width=0.33\textwidth,xshift=0.66\textwidth,anchor=base]{
        \libelle{{\datetxt}} \@date    \\
        \ifdef{\@numero}{%
            \libelle{{\@type\ntxt}} \@numero        
        }{}
        \\
        \ifdef{\@codedest}{%
            \libelle{{\codeclient}} \@codedest        
        }{}
        \ifdef{\@datelimite}{\\\AdvanceDate[\@datelimite]\libelle{{\datelimitetxt}} \@date}{}
    };
    
    \end{tikzpicture}
    

}


\pagestyle{fancy}
\fancyhf{}
\renewcommand{\headrule}{\relax}
\lfoot{\ifnumgreater{\thepage}{1}{\thepage}{}}


\ifsansTVA
    \xdef\tot{}
\else
    \xdef\totTVA{}
    \xdef\totHT{}
    \xdef\totTTC{}
\fi
\newcolumntype{P}[1]{>{\raggedleft}p{#1}}
\newenvironment{facture}{%
    \setlength{\extrarowheight}{0.5ex}
    \setlength{\tabcolsep}{0.5em}
    \arrayrulecolor{encadre}    
    \ifsansTVA%    
        \begin{longtable}{|p{0.2\textwidth}|P{0.2\textwidth}|P{0.2\textwidth}|P{0.2\textwidth}|P{0.2\textwidth}|}%
        \hline%    
        \rowcolor{entetes}\centering\textbf\produit & \centering\textbf\quantite   & \centering\textbf\prixtxt & \centering\textbf\remise & \centering\textbf\tottxt  \tabularnewline[1ex]%
    \else%        
        \begin{longtable}{|p{0.2\textwidth}|P{0.1\textwidth}|P{0.1\textwidth}|P{0.1\textwidth}|P{0.1\textwidth}|P{0.2\textwidth}|P{0.1\textwidth}|}%
        \hline%    
        \rowcolor{entetes}\centering\textbf\produit & \centering\textbf\quantite & \centering\textbf\prixHT & \centering\textbf\remise & \centering\textbf\totHTtxt & \centering\textbf\TVAtxt & \textbf\prixTTC \tabularnewline[1ex]%
    \fi%    
    \endhead%
    \endfoot%
    
    }{%

    \end{longtable}
    
    \begin{tikzpicture}
    \draw[color=white] (0,0) circle (0.001);    
        \node[right,xshift=0.66\textwidth,text width=0.33\textwidth]{%
                        \ifsansTVA%
                            \hfill \libelle{\tottxt:} \numprint[\unite]{\tot}\\%                
                        \else%                
                            \libelle{\totHTtxt:}\hfill \numprint[\unite]{\totHT} \\%
                            \libelle{\totTVAtxt:}\hfill \numprint[\unite]{\totTVA} \\%
                            \libelle{\totTTCtxt:}\hfill \numprint[\unite]{\totTTC} \\
                        \fi
                        \postTotaux};
    \end{tikzpicture}
    
    }

\newcommand{\lignesansTVA}[5]{#1 & #2 & #3 & #4 & #5 \tabularnewline[1ex]}
\newcommandx{\ligne}[5][2=1,4=\tvadefaut,5=0,usedefault]{%
    % 1:produit
    % 2:QT (optionnel)
    % 3:prix HT
    % 4:TVA (optionnel)
    % 5:Remise (optionnel)
    \ifsansTVA
        \fpMul{\prix}{#3}{#2}%
        \fpSub{\prix}{\prix}{#5}%
        \fpAdd{\tot}{\prix}{\tot}%
        \xdef\tot{\tot}%
        \xdef\prix{\prix}%
        %Affichage
        \lignesansTVA{#1}{#2}{\numprint[\unite]{#3}}{\numprint[\unite]{#5}}{\numprint[\unite]{\prix}}%
    \else
        % Prix hors taxe
        \fpMul{\HT}{#3}{#2}%
        \fpSub{\HT}{\HT}{#5}%
        \xdef\HT{\HT}%retenons
        \fpAdd{\totHT}{\totHT}{\HT}%
        \xdef\totHT{\totHT}%
        % Calcul de la TVA
        \fpDiv{\centieme}{#4}{100}%
        \fpMul{\TVA}{\centieme}{\HT}%
        \xdef\TVA{\TVA}%retenons
        \fpAdd{\totTVA}{\totTVA}{\TVA}%
        \xdef\totTVA{\totTVA}%        
        % Prix TTC
        \fpAdd{\TTC}{\HT}{\TVA}%
        \xdef\TTC{\TTC}%
        \fpAdd{\totTTC}{\totTTC}{\TTC}%
        \xdef\totTTC{\totTTC}%    
        % Affichage
        #1 &  #2 &  \numprint[\unite]{#3} & \numprint[\unite]{#5} & \numprint[\unite]{\HT}  & \numprint[\unite]{\TVA} (#4\%) & \numprint[\unite]{\TTC} \tabularnewline[1ex]%
    \fi
    \hline
}
%\end{macrocode}
%    \begin{macrocode}
%    \end{macrocode}
%\Finale

\endinput